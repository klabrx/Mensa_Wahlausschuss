% !TeX root = ../main_nachrueckung_schlichtung_2026.tex


\vspace{1.5\baselineskip}

{
	\setkomafont{section}{\headingfont\bfseries\Huge}
	\section*{Nachbesetzung der zum 01.01.2026 vakant gewordenen Stelle in der Schlichtung}
}

\subsection*{Ausgangslage}
Der Vorstand hat informiert, dass Milena Robbers (M8850) zum 31.12.2025 ihren Austritt aus dem Verein erklärt hat. Damit tritt der in § 3 Abs. 4 der Wahlordnung beschriebene Fall ein:

\begin{rechtszitat}{\S 3 Abs. 4 der Wahlordnung}
	\hoch{1}Scheidet eine in das Amt der Finanzprüfung gewählte Person oder ein Mitglied des Schlichtungsteams vorzeitig aus, so tritt für den Rest der Amtszeit die Person, die bei der letzten Wahl die nächsthöchste positive Stimmendifferenz erreicht hat, an deren Stelle. \hoch{2}Steht eine solche Person nicht zur Verfügung, findet eine Nachwahl nach Maßgabe von § 3 Absatz 3 statt.
\end{rechtszitat}

\subsection*{Bewertung durch den Wahlausschuss}
\begin{itemize}
	\item Milena Robbers wurde bei der turnusmäßigen Wahl 2024 mit einer Amtszeit von 3 Jahren in das Schlichterteam gewählt, ihr Ausscheiden ist somit vorzeitig.
	\item Bei der letzten Wahl (2025) erzielte Ole Oldenburg (M17074) hinter Anna Geiger (M14733) das nächsthöchste positive Ergebnis.
	\item Ole Oldenburg hat dem Wahlausschuss gegenüber erklärt, als Nachrücker zur Verfügung zu stehen.
	\item Eine Nachwahl gem. § 3 Abs. 4 Satz 2 i.V.m. Abs. 3 der Wahlordnung findet somit nicht statt.
\end{itemize}

\subsection*{Feststellung des Wahlausschusses}
Ole Oldenburg (M17074) tritt gem. § 3 Abs. 4 Satz 1 der Wahlordnung die Nachfolge von Milena Robbers (M8850) an. Die Amtszeit läuft bis zur turnusmäßigen Wahl 2027.

\subsection*{Für den Wahlausschuss}
Tina Acham, Andrea Bahrenfuss, Ulf Bahrenfuss, Daniel Bleher, Klaus Brückner, Jo Wilkes