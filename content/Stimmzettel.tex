% !TeX root = ../main.tex


\begin{center}
	{\Huge \headingfont\bfseries Wahlschein zur MV 2026}\\[0.3em]
%	\huge Wahlschein zur MV 2026\\[0.2em]
\end{center}

Pro Amt ist ein Posten zu besetzen. Gemäß Wahlordnung darf auf einem gültigen Wahlschein
für jede kandierende Person pro Amt maximal ein Kreuz bei „Ja“, „Nein“ oder „Enthaltung“
gesetzt werden. Leere Zeilen zählen wie eine Enthaltung für die Person.\\[1em]

% --- AMT 1 ---\includesvg[width=65mm]{../CI/media/wortbildmarke}

\begin{BallotTable}{Vorstand (Vorsitz/Strategie,) für 3 Jahre}
	\CandidateRow{Max Mustermann \quad (M12345)}
	\CandidateRow{Erika Beispiel \quad (M10203)}
	\CandidateRow{--- Platzhalter ---}
\end{BallotTable}

% --- AMT 2 ---
\begin{BallotTable}{Vorstand (Vereinsleben), für drei Jahre}
	\CandidateRow{--- Platzhalter ---}
	\CandidateRow{--- Platzhalter ---}
	\CandidateRow{--- Platzhalter ---}
\end{BallotTable}

% --- AMT 3 ---
\begin{BallotTable}{Schlichtung, für drei Jahre}
	\CandidateRow{--- Platzhalter ---}
	\CandidateRow{--- Platzhalter ---}
\end{BallotTable}

% --- AMT 4 ---
\begin{BallotTable}{Finanzprüfung, für drei Jahre}
	\CandidateRow{--- Platzhalter ---}
	\CandidateRow{--- Platzhalter ---}
	\CandidateRow{--- Platzhalter ---}
\end{BallotTable}
